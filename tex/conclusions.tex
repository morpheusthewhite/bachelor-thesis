\chapter{Conclusioni}
\label{section:conclusion}

L'obiettivo iniziale era quello di sviluppare un'applicazione per l'osservazione, previsione e 
allocazione degli studenti in ambiente accademico: un sistema che riuscisse contemporaneamente a 
monitorare l'afflusso nelle varie lezioni (realizzato attraverso modelli di \textit{face detection} 
in OpenCV), a stimarne l'andamento (attraverso una regressione con una rete neurale in Keras) e 
la distribuzione ottimale (compiuta con la risoluzione di un problema di programmazione matematica
risolto con CPLEX).

\smallskip

Le tecniche utilizzate, che si sono rivelate adatte a questo scopo, offrono però margini di miglioramento:
l'aggiunta di modelli trainati appositamente per la \textit{face detection} in questo ambito migliorerebbe 
ulteriormente la percentuale di volti individuati, arrivando a perfezionare in questo modo le stime,
chiaramente influenzate anche da questi errori; incrementare la dimensione della rete neurale (soprattutto se si 
prendono in considerazione dataset con enorme varietà di entry) risulta quasi necessario, così come 
l'inserimento di ulteriori parametri utilizzati per la stima (come ad esempio il professore che sostiene 
la lezione), che, mancando, limiterebbero la correttezza delle previsioni, semplificando troppo
un fenomeno comunque complesso.

