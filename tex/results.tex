\chapter{Risultati}
\label{section:results}

\section{Rilevamento dei volti}
\label{section:results_fd}

Nel rilevamento dei volti la combinazione dei due modelli pretrainati mostra
una percentuale di rilevamento molto alta, quasi al 100\% nel caso in cui i volti sono 
fotografati frontalmente; i risultati peggiorano invece se il volto è girato oppure 
inquadrato solo in parte. 

\smallskip

\begin{figure}[h]
    \centering
    \begin{subfigure}[b]{0.7\linewidth}
      \includegraphics[width=\linewidth]{detected_3.png}
    \end{subfigure}

    \smallskip
    
    \begin{subfigure}[b]{0.7\linewidth}
        \includegraphics[width=\linewidth]{detected_1.png}
    \end{subfigure}

    \smallskip
    
    \begin{subfigure}[b]{0.7\linewidth}
        \includegraphics[width=\linewidth]{detected_2.png}
    \end{subfigure}
    \caption{Esempi di face detection su immagini di aule con studenti}
    \label{fig:example_opencv}
\end{figure}

\section{Rendimento della rete}
\label{section:results_ml}

Il dataset utilizzato per il training consiste in una serie di entry del 
tipo

\begin{table}[h]
    \begin{small}
        \label{tab:dataset_entry}
        \begin{center}
            \begin{tabular}[c]{c|c|c|c}
                \hline
                \textbf{Giorno} & \textbf{Orario} & \textbf{Materia} & \textbf{Studenti} \\
                \hline
            \end{tabular}
        \end{center}
    \end{small}
\end{table}

\noindent
(ogni entry contiene anche il numero di studenti, trascurando per il momento la 
fase di \textit{face detection}). Esso è generato artificialmente attraverso la generazioni 
di numeri randomici che sono legati tra di loro attraverso dei vincoli matematici che vogliono 
simulare l'andamento reale degli studenti (in modo da avere ad esempio un numero maggiore di 
studenti nelle ore centrali del giorno)

\begin{minted}{python}
dataset = []

# create randomly a dataset
for i in range(DATASET_SIZE):
    subject_index = int(random.random() * len(SUBJECTS))
    hour = int(random.random() * LESSON_HOURS + LESSON_STARTING_HOUR)
    day_index = int(random.random() * len(DAYS))

    # associate to the random choices a number of students
    students = subject_index * 5 - (hour - hour_central)**2 + 100 + day_index % 3 + int(random.random()*16 - 8)
\end{minted}

Il dataset utilizzato contiene 1000 entry, ed il training, a causa dell'
\textit{early stopping} si ferma intorno alla 50 esima iterazione con gli 
score sui set di \textit{training} e \textit{test} rispettivamente mostrate in 
Figure~\ref{fig:scores_training} e Figure~\ref{fig:scores_test}.

\smallskip

\begin{figure}[h]
    \centering
    
    % \begin{subfigure}[b]{0.45\linewidth}
    %     \includegraphics[width=\linewidth]{plots/rmse.png}
    %     \caption{RMSE}
    % \end{subfigure}
    % \begin{subfigure}[b]{0.45\linewidth}
    %     \includegraphics[width=\linewidth]{plots/r_square.png}
    %     \caption{$r^2$}
    % \end{subfigure}
    \begin{subfigure}[b]{0.6\linewidth}
        \includegraphics[width=\linewidth]{plots/loss.png}
    \end{subfigure}
    \caption{La loss (MSE) sul set di training}
    \label{fig:scores_training}
\end{figure}

\begin{figure}
    \centering
    
    \begin{subfigure}[b]{0.45\linewidth}
        \includegraphics[width=\linewidth]{plots/val_rmse.png}
        \caption{RMSE}
    \end{subfigure}
    \begin{subfigure}[b]{0.45\linewidth}
        \includegraphics[width=\linewidth]{plots/val_r_square.png}
        \caption{$r^2$}
    \end{subfigure}
    \begin{subfigure}[b]{0.9\linewidth}
        \includegraphics[width=\linewidth]{plots/val_loss.png}
        \caption{MSE}
    \end{subfigure}
    \caption{Gli score della rete sul set di test}
    \label{fig:scores_test}
\end{figure}

\newpage

\section{Allocazione nelle aule}
\label{section:results_mp}

Avendo fornito come test il seguente set di aule

\begin{table}[h]
    \begin{small}
        \begin{center}
            \begin{tabular}[c]{c|c}
                Nome aula & Capacità \\
                \hline
                r00 & 40 \\
                r01 & 50 \\
                r02 & 60 \\
                r03 & 70 \\
                r04 & 80 \\
                r05 & 90 \\
                r06 & 100 \\
                r07 & 110 \\
                r08 & 120 \\
                r09 & 130 
            \end{tabular}
        \end{center}
    \end{small}
\end{table}

\noindent
i problemi proposti vengono risolti con i seguenti assegnamenti

\begin{table}[h]
    \begin{small}
        \begin{center}
            \begin{tabular}[c]{c|c|c|c|c}
                Giorno & Ora & Disciplina & Stima studenti & Aula assegnata \\
                \hline
                sunday & 15 & SC2 & 86 & r05 \\
                sunday & 15 & PSW & 90 & r06 \\
                sunday & 15 & PFP & 115 & r08 \\
                monday & 10 & SC2 & 84 & r05
            \end{tabular}
        \end{center}
    \end{small}
\end{table}

\newpage

\section{Analisi dei risultati}

I risultati ottenuti sono considerabili molto positivi:

\begin{itemize}
    \item In \textbf{\nameref{section:results_fd}} si nota come anche solo con due modelli pretrainati
        computazionalmente poco onerosi si sia raggiunto un rilevamento quasi totale dei volti
    \item In \textbf{\nameref{section:results_ml}} gli score, ottenuti da una rete neurale molto semplice,
        evidenziano, in particolare il RMSE, come l'errore nelle stime effettuate sia circa 
        di 5 studenti, circa il 5\% del numero medio di studenti del dataset
    \item In \textbf{\nameref{section:results_mp}} gli assegnamenti effettuati dall'applicazione rispettano
        completamente le richieste precedentemente specificate: altre soluzioni ammissibili per il problema,
        come ad esempio 

        \begin{table}[h]
            \begin{small}
                \begin{center}
                    \begin{tabular}[c]{c|c|c|c|c}
                        Giorno & Ora & Disciplina & Stima studenti & Aula assegnata \\
                        \hline
                        sunday & 15 & SC2 & 86 & r06 \\
                        sunday & 15 & PSW & 90 & r07 \\
                        sunday & 15 & PFP & 115 & r09 \\
                        monday & 10 & SC2 & 84 & r06
                    \end{tabular}
                \end{center}
            \end{small}
        \end{table}

        risultano avere un valore della funzione obiettivo inferiore.

    \end{itemize}
